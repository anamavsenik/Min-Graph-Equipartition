\documentclass[12pt,a4paper]{amsart}
\usepackage[slovene]{babel}
\usepackage[utf8]{inputenc}
\usepackage{amsmath,amssymb,amsfonts}
\usepackage{url}
\usepackage[dvipsnames,usenames]{color}


\textwidth 15cm
\textheight 24cm
\oddsidemargin.5cm
\evensidemargin.5cm
\topmargin-5mm
\addtolength{\footskip}{10pt}
\pagestyle{plain}
\overfullrule=15pt 
% ukazi za matematicna okolja
\theoremstyle{definition} % tekst napisan pokoncno
\newtheorem{definicija}{Definicija}[section]
\newtheorem{primer}[definicija]{Primer}
\newtheorem{opomba}[definicija]{Opomba}

\renewcommand\endprimer{\hfill$\diamondsuit$}

\theoremstyle{plain} % tekst napisan posevno
\newtheorem{lema}[definicija]{Lema}
\newtheorem{izrek}[definicija]{Izrek}
\newtheorem{trditev}[definicija]{Trditev}
\newtheorem{posledica}[definicija]{Posledica}

% za stevilske mnozice uporabi naslednje simbole
\newcommand{\R}{\mathbb R}
\newcommand{\N}{\mathbb N}
\newcommand{\Z}{\mathbb Z}
\newcommand{\C}{\mathbb C}
\newcommand{\Q}{\mathbb Q}

% ukaz za slovarsko geslo
\newlength{\odstavek}
\setlength{\odstavek}{\parindent}
\newcommand{\geslo}[2]{\noindent\textbf{#1}\hspace*{3mm}\hangindent=\parindent\hangafter=1 #2}


\newcommand{\imeavtorja}{Ana Marija Kravanja}
\newcommand{\naslovdela}{Min-Graph Equipartition Problem with Simulated Annealing}
\newcommand{\letnica}{2019} 

\begin{document}

% od tod do povzetka ne spreminjaj nicesar
\thispagestyle{empty}
\noindent{\large
UNIVERZA V LJUBLJANI\\[1mm]
FAKULTETA ZA MATEMATIKO IN FIZIKO\\[5mm]}
\vfill

\begin{center}{\large
{\bf \naslovdela}\\[10mm]
Ana Marija Kravanja, Urška Jeranko, Oskar Kregar}\\[1cm]

\end{center}
\vfill

\noindent{\large
Ljubljana, \letnica}
\pagebreak


\section{Osnovno o projektu}
Reševali bomo problem deljenja grafa. Graf želimo razdeliti na dva priližno enaka dela pri čemer je med njima čim manj povezav.  Reševali bomo s požrešno metodo in metodo simuliranega izničenja. Rešitve teh metod bomo primerjali med sabo in jih preizkusili na več različnih grafih pri različnih parametrih. 

\section{Požrešna metoda}
Požrešna metoda je strategija pri kateri optimalno rešitev izberemo na vsakem koraku posebej s ciljem da bi našli globalno optimalno rešitev. Problem metode je, da na vsakem koraku izberemo najboljšo rešitev, kar nas pripelje do lokalnega optimuma, ne pa nujno globalnega. Možno je tudi, da nas privede celo do najslabšega globalnega rezultata. \\

Naj bo $G=(V,E)$ enostaven graf, pri čemer je $V$ množica vozlišč in $E$ množica povezav. Naj bo število vozlišč enako $n$. Definirajmo delitev grafa na dve množici X in Y, pri čemer je $|X| = \lceil \frac{n}{2} \rceil$. Širina bisekcije je najmanjše število povezav med $X$ in $Y$ med vsemi možnimi delitvami. 

\begin{definicija}
Naj bo $G=(V,E)$ enostaven graf in $X,Y \subseteq V$, tako da je $X \cap Y = \emptyset$ in $X \cup Y =V$.
\begin{itemize}
\item Za $x \in X$ označimo $I(x)$ notranja vrednost, to je šteilo povezav $(x,z) \in E;z\in X \backslash \{x\}$ . Analogno definiramo $I(y)$ za $y \in Y$.
\item Za $x \in X$ označimo $O(x)$ zunanja vrednost, to je šteilo povezav $(x,z) \in E;z\in Y $ . Analogno definiramo $O(y)$ za $y \in Y$.
\item Za $x \in X, y \in Y$ naj bo $\omega(x,y) := \begin{cases} 1,&\text{if} (x,y) \in E\\ 
0, &\text{sicer}\end{cases} $.
\item Za $x \in X, y \in Y$ naj bo $S(x,y):= O(x)-I(x)+O(y)-I(y)-2\omega(x,y)$.
\end{itemize}
\end{definicija}

\subsection{Algoritem požrešne metode}
Začnemo z naključno delitvijo vozlišč grafa. Algoritem zamenja dve vozlišči na različnih straneh, če nam to predstavlja boljšo bisekcijo in se ustavi, če to ni več možno. Pri menjavi notranje vrednosti postanejo zunanje in obratno, zato dobimo izboljšano bisekcijo le če je $S(x,y)>0$.
\subsubsection{Algoritem 1} 
Vhodni podatki: Graf $G=(V,E), |V|=n$.
\begin{enumerate}
\item Izberemo neključno delitev $(X,Y)$.
\item Izberemo $x\in X, y\in Y$, tako da je $S(x,y)>0$.
\item Zamenjamo vozlišči $x$ in $y$.
\item Ponavljamo 2. in 3. korak dokler ne obstajata več $x\in X, y\in Y$, da je $S(x,y)>0$.
\end{enumerate}
Izhodni podatki: delitev $(X,Y)$.

\section{Metoda simulacijskega izničenja}
Metoda simulacijskega izničenja je verjetnostna strategija za aproksimacijo globalnega optimuma dane funkcije. Razlika od požrešne metode je to, da na vsakem koraku lahko izberemo tudi slabšo rešitev z neko verjetnostjo, vendar nas na koncu privede do globalnega optimuma. 

\subsection{Algoritem metode simulacijskega izničenja}
Graf si definiramo analogno kot pri požrešni metodi. Če je $S(x,y)>0$ zamenjamo vozlišči, če pa je $S(x,y)<0$ pa z verjetnostjo $$P(x,y,t) = e^{\frac{S(x,y)}{t}}; t\ge 0$$.

\subsubsection{Algoritem 2}
\begin{enumerate}
\item t= temperatura z visoko vrednostjo
\item (X,Y) = začetna delitev grafa
\item N= (X,Y)
\item ponovi dokler je N najboljša rešitev, nam je zmanjkalo časa ali pa je $t \leq 0$
\item 		za vsak $x, y \in (X,Y)$
\item 			D'=delitev, kjer zamenjamo $x$ in $y$
\item 			če je $S(x,y) >0$ ali naključno število med 0 in 1 manjše od $P(x,y,t)$, potem je $D'=(X,Y)$
\item 			zmanjšamo t
\item 			če je $S(x,y)$		


\end{enumerate}

\end{document}
